\documentclass[]{article}
\usepackage{lmodern}
\usepackage{amssymb,amsmath}
\usepackage{ifxetex,ifluatex}
\usepackage{fixltx2e} % provides \textsubscript
\ifnum 0\ifxetex 1\fi\ifluatex 1\fi=0 % if pdftex
  \usepackage[T1]{fontenc}
  \usepackage[utf8]{inputenc}
\else % if luatex or xelatex
  \ifxetex
    \usepackage{mathspec}
  \else
    \usepackage{fontspec}
  \fi
  \defaultfontfeatures{Ligatures=TeX,Scale=MatchLowercase}
\fi
% use upquote if available, for straight quotes in verbatim environments
\IfFileExists{upquote.sty}{\usepackage{upquote}}{}
% use microtype if available
\IfFileExists{microtype.sty}{%
\usepackage{microtype}
\UseMicrotypeSet[protrusion]{basicmath} % disable protrusion for tt fonts
}{}
\usepackage[unicode=true]{hyperref}
\hypersetup{
            pdfborder={0 0 0},
            breaklinks=true}
\urlstyle{same}  % don't use monospace font for urls
\IfFileExists{parskip.sty}{%
\usepackage{parskip}
}{% else
\setlength{\parindent}{0pt}
\setlength{\parskip}{6pt plus 2pt minus 1pt}
}
\setlength{\emergencystretch}{3em}  % prevent overfull lines
\providecommand{\tightlist}{%
  \setlength{\itemsep}{0pt}\setlength{\parskip}{0pt}}
\setcounter{secnumdepth}{0}
% Redefines (sub)paragraphs to behave more like sections
\ifx\paragraph\undefined\else
\let\oldparagraph\paragraph
\renewcommand{\paragraph}[1]{\oldparagraph{#1}\mbox{}}
\fi
\ifx\subparagraph\undefined\else
\let\oldsubparagraph\subparagraph
\renewcommand{\subparagraph}[1]{\oldsubparagraph{#1}\mbox{}}
\fi

% set default figure placement to htbp
\makeatletter
\def\fps@figure{htbp}
\makeatother


\date{}

\begin{document}

\section{Behavioral Economics: Libertarian Paternalism and Public
Health}\label{behavioral-economics-libertarian-paternalism-and-public-health}

\subparagraph{Canon - Applied Economic Anaylsis I, Tilburg School of
Economics and
Management}\label{canon---applied-economic-anaylsis-i-tilburg-school-of-economics-and-management}

By Yvonne Jamar, Carolina Sant'Ana Oliveira, Sandra van der Schaaf,
Florian Schmidt, Zeynep Tanca and Francesca Ventimiglia

\paragraph{ABSTRACT}\label{abstract}

\emph{Public health is a matter of great concern for politicians and
legislators around the world: not only does it take up a relevant slice
of a nation's public budget, but it also deeply affects citizens'
quality of life. The behavioural economics approach of libertarian
paternalism largely contributes to the already substantial literature on
this matter. In the following canon, we give an overview of the
aforementioned contributions by presenting the most important examples
of practical applications. These include experiments on nudging people
towards healthier lifestyles as well as articles on better schemes for
organ donation and topics as the improvement of hygiene in hospitals.}

\paragraph{INTRODUCTION}\label{introduction}

How can we make people quit smoking? Which way is the best to increase
the number of organ donators? Why are so many people unable to cope with
choices concerning their health -- and is it possible to improve their
decisions? These questions are just a few out of many politicians and
health managers permanently face when considering new regulations and
policies regarding the system of public health. Doubtlessly, there are
plenty of ways to answer them. The paper at hand tries to give an
insight how the approach of Behavioural Economics and Libertarian
Paternalism can help to deal with these issues.

Libertarian Paternalism is a term introduced by US economists Richard
Thaler and law professor Cass Sunstein in their well-known bestseller
``Nudge -- Improving decisions about health, wealth and happiness''
(2008). This book is one of the most influential works regarding
Behavioural Economics, a new stream of economic theory taking into
account that most people do not behave totally rational but also
emotional and sometimes irrational. The general idea of Libertarian
Paternalism is, that it is possible to improve people's (sometimes poor)
choices by nudging (i.e.~softly push) them towards a (socially)
desirable behaviour whilst still letting them the freedom to decide
differently if they want so.

On the following pages we want to give an overview of how this approach
can be applied to take care of public health. Summarising studies and
papers of four applications, we show that there are many ways to improve
public health by the insights of Behavioural Economics and that most of
them are way easier to impose and cost-saving than many people expect.

This canon is divided into four chapters. It commences with a brief
overview of the influence of manipulative advertisement and how the
state can nudge people towards a healthier lifestyle. In the second
chapter we discuss one of the most classic examples of Behavioural
Economics and public health -- the possibility to increase organ
donations by imposing a default option so that every citizen is
automatically an organ donator. Following to that we show what impact
monitoring of hospital employees on hygiene can have and how patient's
choices at the doctors can be improved. Eventually we emphasise on the
positive nudging effect smartphone apps and other devices can have to
promote healthier lifestyles and choices.

\paragraph{NUDGING TOWARDS A HEALTHIER
LIFESTYLE}\label{nudging-towards-a-healthier-lifestyle}

The traditional persuasion messages are starting to show their limited
impact on health decisions and therefore, insights from behavioural
economics are becoming more and more relevant in healthcare management
(Voyer, 2015).

One of the best known health related examples of how people can be
nudged is encouraging them to quit smoking. It is commonly known that
smoking is an unhealthy habit. And even though the taxes on harmful
substances such as tabacco keep increasing, many smokers have been
unable to quit (Matjasko, Cawley, Baker-Goering \& Yokum, 2016). Parkes,
Greenhalgh, Griffin and Dent (2008) conducted an experiment where they
used the lung age of smokers as an incentive for them to quit. Lung age
was defined as the age of the average healthy participant whose results
of the long test was equal to the result of the smoker. The authors
found that emphasizing the lung age, rather than the raw results of the
test, significantly increased the probability for smokers to quit. This
result might me related to the effect of social norms. Blumenthal-Barby
and Burroughs (2012) mention the ``Charm Project'' in the UK and the
``Most of Us Wear Seatbelts'' campaign in Montana. Respectively,
lifestyle and exercise choices of participants were compared with those
of peers to incentivize them towards a healthier lifestyle, and people
were reminded that 85\% of the population wear seatbelts, which resulted
into a significant increase in the reported use of seatbelts.

Another method that was found to be effective in making people stop
smoking is through salience and affect (Blumenthal-Barby \& Burroughs,
2012). Participants were shown a video in which they or one of their
loved ones suffered from a heart attack. Three months later, 50\% of the
participants claimed to have quit smoking. This effect is based on
people being sensitive to novel, personally relevant, or vivid examples
and explanations. Other health enhancing applications of this method are
depicturing food that corresponds to the amount of calories burned while
exercising and stating the amount of calories per dish on menus.

Unhealthy food consumption can also be reduced by priming
(Blumenthal-Barby \& Burroughs, 2012). This method makes use of
subconscious cues that influence the decision making process. For
instance, in a school cafeteria, fruit consumption went up 54\% when it
was displayed attractively and placed prominently. And, after handing
out smaller food containers in an all-you-can-eat environment, people
will consume less. Simply asking people about their health habits, like
flossing or eating fast food, nudged them toward healthier choices as
well. Another kind of primer is a reminder to make healthier choices.
Examples are reminding people to exercise and plan their screening
appointments (Blumenthal-Barby \& Burroughs, 2012). Matjasko et al.
(2016) state the importance of framing. That is, framing messages about
a certain behaviour or treatment in a way that emphasizes the positive
effect on health are more effective when it concerns the prevention of
health issues. When a message is framed in such a way that the negative
effects on health are emphasized, screening behaviour is encouraged.
Blumenthal-Barby and Burroughs (2012) found a higher effectiveness of
messages aimed at young adults, when they feel similar to the messenger.

Another possibility is to supply people who have time-inconsistent
preferences with commitment devices (Matjasko et al., 2016). Such
consumers find themselves giving into temptations, instead of living the
healthy lifestyle they desire. A commitment device proposed by Richard
Thaler is a deposit contract, which will only give back the deposit if
the predetermined goal is reached. Research has found that the
completion of more weeks of exercise can be accomplished by nudging
people to make commitments for a longer term (Matjasko et al., 2016).

\paragraph{OPT-IN VS OPT-OUT}\label{opt-in-vs-opt-out}

Nowadays, in almost every supermarket chain, cashiers do not offer a
plastic bag to customers when they check out. Customers in need of a bag
are usually forced to ask for one and/or to pay for it. In such
scenarios, customers are less likely to explicitly ask for a shopping
bag, as the default option, the one suggested by the cashier's
behaviour, i.e.~no plastic bag, is the status quo. Hence, it would
require some effort to deviate from the norm entailed in such a stable
situation and losing the status quo by buying a plastic bag might not be
worth the gain (not having to carry the groceries by hand) (Samson,
2014). This example on daily life experience perfectly summarizes the
concept of default options: rules determining the set of actions to be
undertaken if the decision-maker fails to establish her/his preferred
choice in a certain situation. As already mentioned in the introduction,
this section is dedicated to the leading and most publicly recognized
application of default options: Johnson and Goldstein's article on the
impact of default choices on organ donation compliance rates.

The article presents three different studies on the effects of opt-in,
opt-out programs (default choices) and programs that do not entail any
default choice on organ donations. All of these studies show that
opt-out programs, i.e.~programs in which one has to specifically
register to stop being an organ donor, positively affect organ donations
by significantly raising the rate of compliance.

In particular, the authors carry out an analysis of the differences in
organ donations among European countries that is of incredible
relevance. Within Europe, the differences in organ donations are
consistent even when checking for relevant economic, social and
political differences across nations. However, countries in the Vieux
Continent show different approaches to the so called ``no-action
default'' in organ donation, i.e.~what action to apply as a default when
the citizen does not specify her/his own will. In detail, two mechanisms
are applied: states applying the explicit-consent expect their citizen
to specifically register to be organ donors, while presumed-consent
countries consider their citizen organ donors unless they register not
to be one. In the article, almost 60 percentage points separate the low
rates of compliance to organ donation in countries where the government
applies an opt-in (explicit-consent) mechanism from those countries
where an opt-out (presumed-consent) mechanism was established. One of
the studies dealing with hand hygiene compliance of the staff in
hospitals is a study conducted by Armellino et al (2013). This study
looks at the effect of using RVA (Remote Video Auditing) in the surgical
intensive care units and feedback on the hand hygiene compliance rates
of the hospital staff. During a 4-week period of monitoring the staff
via RVA, the hand hygiene compliance rate on average was 30.42\%. After
the 4th week, real time feedback was combined with RVA and in this
post-feedback period hand hygiene compliance rates exceeded 80\% on
average. It's concluded that RVA combined with feedback generated a
significant and sustained improvement in hand hygiene compliance rates
(Armellino et al., 2013). A very similar study conducted by Walker et
al. (2014) assessing the effectiveness of a new hand hygiene monitoring
program also found out that the program showed significant increase in
hand hygiene compliance. In addition to the compliance in hand hygiene,
Overdyk et al. (2015) carried out a study that evaluated the effect of
remote video auditing together with real time feedback on the compliance
with the safety processes such as surgical checklist. They too concluded
that improved efficiency and compliance to safety protocols can result
from direct observation, measurement and immediate feedback to the
operation team (Overdyk et al., 2015).

Another issue where feedback mechanisms are used in healthcare is
unattended appointments in hospitals. When patients do not keep up with
their appointments, this has serious costs for the hospital, since this
causes significant losses in revenues and decrease in the service
quality and patient satisfaction. A study conducted by Martin, Bassi and
Dunbar-Rees (2012), looking at the effect of certain methods on reducing
the number of missed appointments, concludes that the exposure to social
norms as the number of appointments that were attended, proved to be the
most effective method among others. This was basically done by
communicating on posters or on screens the number of patients who kept
up with their appointments in the previous periods. The exposure to
right norms functions in a way similar to feedback mechanisms and it
indeed reduced the number of missed appointments by 31.7\% (Martin et
al., 2012).

\paragraph{GAMIFICATION}\label{gamification}

With the outstanding use of smartphones, tablets, laptops and all
devices that provide internet, it is not hard to imagine that all kind
of services would be available in the e-world. One of the most important
services that has been influenced by the internet is the health care
area. A new term was raised and is now becoming very popular; it is the
e-health, which characterizes everything related to health care and
computers. The internet has revolutionized the traditional healthcare,
opening several opportunities to consumers, firms and governments to
benefit from it. Never before consumers were advertised with so many
information about health care and medicines. Consumers now have the
chance to learn more in an easier and faster way. Firms and health
insurances can benefit from it, receiving more feedbacks and acquiring
more consumers. In addition, governments have now the possibility to use
new approaches in order to stimulate healthier life styles of its
citizens.

A promising application of Behavioural Economics that involves both
computer devices and healthcare is the use of gamification of e-health.
Gamification means the use of games on non-games situations, also known
as health-monitoring devices when concerned to health. These devices can
be found in distinct ways. They can provide information about users'
training, functioning as speed sensor, calorie counter, heart rate
monitor, etc. They can also work as medical laboratory at home,
measuring the level of glucose, level of testosterone, Vitamin D, etc.
Going beyond the provision of information about users, these devices can
inform about the quality of the air, which ingredients are in each meal,
whether a food is fresh or not, etc. In addition, health-monitoring
devices can create competitions among users, making it possible for
users to share their training performances and goals. In other words,
through these new devices, for instance FitBit, Garmin and Apple Watch,
users have better and more ways to monitor their health status, having
real-time feedbacks about their trainings and health behaviours.

From the entire possibilities that resulted from gamification, it is
possible to incorporate several Behavioural Economics principles. By
means of monitoring and real-time feedback, the users can be nudged
towards better decisions. The feeling of being observed and the
stimulation of receiving real-time feedback change the users' behaviour.
Furthermore, users very well accept health-monitoring devices, once it
provides them with short and long-term objectives, motivating and
increasing users' commitment.

A much-mentioned study in this field is Smyth and Heron (2010). The
authors aim to present and criticize the use of mobile technology to
improve health behaviours and psychological and physical symptoms. The
study consist of a base of twenty-seven mobile technology interventions
to deliver ambulatory treatment on patients with smoking cessation,
weight loss, anxiety, diabetes management, eating disorders, alcohol
use, healthy eating and physical activity. The article emphasizes that
mobile-technology devices have a positive effect on treating health
behaviours and that the patients very well accepted them.

The use of gamification of e-health is indeed one of the biggest
promises of the healthcare sector. With the rise of healthcare
information and the constant growth of the use of smartphones, the
acceptance of gamification tends to be even higher than it is nowadays.
Moreover, the youth claims to a fun and dynamic relationship between
health and health devices. \#\#\#\#\# REFERENCES

Armellino, D. et al. (2013). Replicating changes in hand hygiene in a
surgical intensive care unit with remote video auditing and feedback.
\emph{American Journal of Infection Control}, 41(10), 925-927

Blumenthal-Barby, J.S., \& Burroughs, H. (2012). Seeking better health
care outcomes: the ethics of using the ``nudge''. \emph{The American
Journal of Bioethics}, 12(2), 1-10

Haynes S.N., \& Horn W.F. (1982). Reactivity in behavioral observation:
A review. \emph{Journal of Behavioral Assessment}, 4(4), 369-385

Heron, K.E., \& Smyth, J.M. (2010). Ecological momentary interventions:
incorporating mobile technology into psychosocial and health behaviour
treatments. \emph{British journal of health psychology}, 15(1), 1-39

Johnson, E.J., \& Goldstein, D. (2003). Do Defaults Save Lives?.
\emph{Science}, 302(5649), 1338-1339

Martin, S.J., Bassi, S., \& Dunbar-Rees, R. (2012). Commitments, norms
and custard creams -- a social influence approach to reducing did not
attends (DNAs). \emph{Journal of the Royal Society of Medicine}, 105(3),
101-104

Matjasko, J.L., Cawley, J.H., Baker-Goering, M.M., \& Yokum, D.M.
(2016). Applying Behavioral Economics to Public Health Policy:
Illustrative Examples and Promising Directions. \emph{American Journal
of Preventive Medicine}, 50(5 suppl 1), 1-12

Orlowska, D. (1990). Staff reactivity to observation. \emph{The British
Journal of Mental Subnormality}, 36(71), 125-131

Overdyk, F.J. et al. (2015). Remote video auditing with real-time
feedback in an academic surgical suite improves safety and efficiency
metrics: a cluster randomised study. \emph{BMJ quality \& safety}, 0,
1-7

Parkes, G., Greenhalgh, T., Griffin, M., \& Dent, R. (2008). Effect on
smoking quit rate of telling patients their lung age: the Step2quit
randomised controlled trial. \emph{British Medical Journal}, 336(7644),
598-600

Samson, A. (2014, February 25, ). \emph{A simple change that could help
everyone drink less}. Retrieved from
https://www.psychologytoday.com/blog/consumed/201402/simple-change-could-help-everyone-drink-less

Sykes, R. (1978). Toward a theory of observer effect in systematic field
observation. \emph{Human Organization}, 37(2), 148-156

Thaler, R. H., \& Sunstein, C. R. (2008). \emph{Nudge: Improving
Decisions About Health, Wealth, and Happiness}. Yale University Press

Voyer, B.G. (2015). `Nudging' behaviours in healthcare: Insights from
behavioural economics. \emph{British Journal of Healthcare Management},
21(3), 130-135

Walker, J.L. et al. (2014). Hospital hand hygiene compliance improves
with increased monitoring and immediate feedback. \emph{American Journal
of Infection Control}, 42(10), 1074-1078

\end{document}
